\section{Experiments}
\label{sec:expts}
%
In all our experiments, robust cohesion and anisotropy is reinforced by having one player be a pursuer against a flock of evaders as seen in \autoref{fig:robust_heading}. We initialize each flock's agents to distinct positions on the vectogram. All agents share the same linear speed, $v$ but varying orientations $\langle w^{(i)} (t) \rangle_r$ that are averaged across every nearest neighbor in a flock according to \eqref{eq:DubinsJadbabaie}. Nearest neighbors are updated according to Algorithm \ref{alg:neighbors}.
%

%\begin{figure}[tb!]
%	\centering
%	%
%	\begin{minipage}[b]{.45\textwidth}
%		\includegraphics[width=1.1\textwidth, height=1.4\textwidth]{figures/payoff_single.jpg}
%	\end{minipage}
%		%
%	\begin{minipage}[b]{.45\textwidth}
%		\includegraphics[width=1.\textwidth, height=1.4\textwidth]{figures/flock_1.jpg}
%	\end{minipage}
%	\footnotesize{\caption{Left:Initial zero level RCBRT of a single agent's payoff within a flock.   Right: Avoid zero level RCBRT of the aggregated payoffs of all  agents that constitute an arbitrary flock in our setup.  (Metric reach radius=$0.2m$, Avoid Radius=$0.2m$).}}
%	\label{fig:starlings}
%\end{figure}
%
At issue is a family of games based on different starting points for local flocks that on the whole constitute a murmuration. Therefore, we pre-define payoffs for each individual agent in every flock implicitly on a flock's state space as $\valuefunc: [-T, 0]\times \mc{X} \rightarrow \bb{R}$. 
%
In starlings, agents move in local flocks of six to seven nearest neighbors~\cite{Ballerini1232} in order to preserve cohesion and heading consensus. Therefore, we define the target set and the tube as
%
\begin{align}
	&\targetset_0 = \left\{ \state \in \bar{\openset} | \valuefunc(\state, 0) \le 0, \valuefunc(\state, 0) = \valuefunc_1(\state_1, 0) \cup \cdots \cup \valuefunc_n(\state_n, 0)\right\} , \\
	&\mathcal{L}([\tau, 0],  \mathcal{L}_0) = \left\{\state\in \bar{\openset}  | \valuefunc(\state, \tau) \le 0, \valuefunc(\state, 0) = \valuefunc_1(\state_1, 0) \cup \cdots \cup \valuefunc_n(\state_n, 0)\right\} \nonumber
\end{align}
%
where $\tau \in  [-T, 0]$. A pictorial representation of the zero-level RCBRT of a differential  game with six-seven agents in the evading flock (\cf\autoref{fig:robust_heading}), constructed from a union of each agent's respective payoff,  is depicted in \autoref{fig:flocks_multi}. Each agent's payoff is implicitly defined by a signed distance representation on the state space~\cite{LevelSetsBook}. Abusing notation and dropping the $i$th superscript for an agent, we construct $\valuefunc(\cdot)$ as 
%
\begin{align}
	\valuefunc(\state, 0) = \sqrt{\state_1^2 + \state_2^2} - r_c
\end{align}
%
where $r_c$ is the capture radius, equivalent to the topological range for a flock as reported in~\cite{Ballerini1232}. 

\begin{figure}[tb!]
	\centering
	\begin{tabular}{ccc} 
		\includegraphics[height=12em,width=10em]{figures/flock_1.jpg} 
		&
		\includegraphics[height=12em,width=10em]{figures/flock_2.jpg} 
		& 
		\includegraphics[height=12em,width=10em]{figures/flock_3.jpg} 
		\\	
		\includegraphics[height=12em,width=10em]{figures/flock_4.jpg}
		&
		\includegraphics[height=12em,width=10em]{figures/flock_5.jpg} 
		&
		\includegraphics[height=12em,width=10em]{figures/flock_6.jpg}
	\end{tabular}
	\caption{\footnotesize Illustrative Initial Zero-Level RCBRT for Different Flocks in our Proposal: Avoid RCBRT of the aggregated payoffs of all  agents that constitute each arbitrary flock that constitute a murmuration in our setup.  (Metric reach radius=$0.2m$, Avoid Radius=$0.2m$).} 
	\label{fig:flocks_multi}
\end{figure}

\subsection{Interagents Spatial Structure}

Every local flock has its own payoff, whose target set, together with those of nearest neighbors being interacted with are related by the surfaces. The zero level set of the union of these payoffs constitute the avoid set for a heading consensus. To ensure adequate spatial separation between every agent, we initialize a flock's $j$'s agents, $i=1,\ldots, n$ on the vectogram in the following way:
%
\begin{align}
	\state^{(i)_j} = \left[\begin{array}{ccc}
		r_c \cos (\frac{5\pi}{2}\frac{i}{6}), & 	r_c \sin (\frac{5\pi}{2}\frac{i}{6}), & H+i\,\delta H
	\end{array}\right] \, \text{for } i=1, \cdots, n, H=0.1, \delta H = 0.05.
\end{align}
