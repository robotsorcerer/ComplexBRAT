\section{Experiments}
\label{sec:expts}
%
At issue is a family of games based on different starting points for local flocks that on the whole constitute a murmuration (see \autoref{fig:flocks_multi}). 
In starlings, agents move in local flocks of six to seven nearest neighbors~\cite{Ballerini1232} in order to preserve cohesion and heading consensus.  Therefore, we implicitly compute the payoff for each individual agent in every flock. In this light, a flock's payoff is a union (element-wise minimum f respective payoff points) of the payoff of every agent within it. 
%
\begin{figure}[tb!]
	\centering
	\begin{tabular}{ccc} 
		\includegraphics[height=12em,width=10em]{figures/flock_0_init.jpg} 
		&
		\includegraphics[height=12em,width=10em]{figures/flock_1_init.jpg} 
		& 		
		\includegraphics[height=12em,width=10em]{figures/flock_2_init.jpg} 
		\\
		\includegraphics[height=12em,width=10em]{figures/flock_0_final.jpg} 
		&
		\includegraphics[height=12em,width=10em]{figures/flock_1_final.jpg} 
		& 		
		\includegraphics[height=12em,width=10em]{figures/flock_2_final.jpg} 		
	\end{tabular}
	\caption{\footnotesize Top: Initial zero-level set for a flock initialized from a specific initial condition.  Bottom: Interface of the respective evading flock under attack from a pursuer at 
		the end a Lax-Friedrichs' integration.   (Metric reach radius=$0.2m$, Avoid Radius=$0.2m$). More results are included in the appendix \ref{sec:rcbrats}.} 
	\label{fig:flocks_multi}
\end{figure}
%
Therefore, we define the target set and the tube as
%
\begin{align}
	&\targetset_0 = \left\{ \state \in \bar{\openset} | \valuefunc(\state, 0) \le 0, \valuefunc(\state, 0) = \valuefunc_1(\state_1, 0) \bigcup \cdots \bigcup \valuefunc_n(\state_n, 0)\right\} , \\
	&\mathcal{L}([\tau, 0],  \mathcal{L}_0) = \left\{\state\in \bar{\openset}  | \valuefunc(\state, \tau) \le 0, \valuefunc(\state, 0) = \valuefunc_1(\state_1, 0) \bigcup \cdots \bigcup \valuefunc_n(\state_n, 0)\right\} \nonumber
\end{align}
%
where $\tau \in  [-T, 0]$. 

%\begin{figure}[tb!]
%	\centering
%	\begin{minipage}[b]{.45\textwidth}		
%		\includegraphics[width=\textwidth]{figures/murmurations_0000.jpg} 
%	\end{minipage}
%	%	
%	\begin{minipage}[b]{.45\textwidth}		
%		\includegraphics[width=\textwidth]{figures/murmurations_0100.jpg} 
%	\end{minipage}
%	%
%	\\
%	%
%	\begin{minipage}[b]{.45\textwidth}		
%		\includegraphics[width=\textwidth]{figures/murmurations_0250.jpg} 
%	\end{minipage}
%	%
%	\begin{minipage}[b]{.45\textwidth}		
%		\includegraphics[width=\textwidth]{figures/murmurations_0480.jpg} 
%	\end{minipage}
%	%	
%	\caption{\footnotesize RCBRT of an Evading Flock-Pursuer System at Different Lax-Friedrichs' integration Timesteps.   
%		(Metric reach radius=$0.2m$, Avoid Radius=$0.2m$. Topological radius=3 units).} 
%	\label{fig:flocks_rcbrt}
%\end{figure}
%
A pictorial representation of the zero-level robustly controllable backward reach-avoid tubes (RCBRAT) of various differential  games with finite agents in the evading flock (\cf\autoref{fig:robust_heading})\footnote{More results for other flocks are included in the appendix.}, constructed from a union of each agent's respective payoff,  is depicted in \autoref{fig:flocks_multi}. These are  the aggregated payoffs of all  agents that constitute each  flock. Each agent's payoff is implicitly defined by a signed distance representation on the state space~\cite{LevelSetsBook} \ie, %. Abusing notation and dropping the $i$th superscript for an agent, we construct $\valuefunc(\cdot)$ as 
%
\begin{align}
	\valuefunc^{(i)}(\state, 0) = \sqrt{{\state_1^{(i)}}^2 + {\state_2^{(i)}}^2} - r_c^{(i)}
\end{align}
%
where $r_c$ is the capture radius. And the payoff for an evading flock is taken as the union of all individual agent's payoff as
%
\begin{align}
	\valuefunc_j(\state, 0) = \bigcup_{i=1}^{n_a} \sqrt{{\state_1^{(i)}}^2 + {\state_2^{(i)}}^2} - r_c^{(i)} 
\end{align}

In the original fronts propagation algorithm of \cite{OsherFronts}, the zero-level set in embedded throughout the state space. This adds a lot of computational complexity to the computational procedure. To circumvent this, in each iteration of the numerical integration scheme, we compute the grid bounds of the zero-level set of an interface, and we update the state space with the newly constructed grid. This hastens computational time, by reducing the complexity of the numerical flux integration scheme to just the number of elements on the front. 

%\subsection{Interagents Spatial Structure}

Every local flock has its own payoff, whose target set, together with those of nearest neighbors being interacted with are related by the surfaces. The zero level set of the union of these payoffs constitute the avoid set for a heading consensus. To ensure adequate spatial separation between every agent, we initialize a flock's $j$'s agents, $i=1,\ldots, n$ on the vectogram in the following way:
%
\begin{align}
	\state^{(i)_j} = \left[\begin{array}{ccc}
		r_c \cos (\frac{i\pi}{4}), & 	r_c \sin (\frac{i\pi}{4}), & h+i\,\delta h
	\end{array}\right]^T \, %\text{for } i=1, \cdots, n, 
\end{align}
%
where $h=0.1, \, \delta h = 0.05$ and $r_c$ is a collision avoidance radius.

We initialize each flock's agents to distinct positions on the vectogram. All agents share the same linear speed, $v$ and their orientations, $\langle w^{(i)} (t) \rangle_r$, are averaged  across every agent that falls within a nearest neighboring radius, $r_i$ of the label of agent $i$. This is expressed on line \ref{alg:neighbors:line_neigh_rad} of Algorithm \ref{alg:neighbors} . in a flock according to \eqref{eq:DubinsJadbabaie}. Nearest neighbors are updated as described in Algorithm \ref{alg:neighbors}. 

We want an adaptive allocation rule for the pursuer when aiming for an evading individual agent since in the process of hunting for a prey, an originally targeted prey may evade an attacker. We want the attacker to resort to redundant options so that anisotropy is enforced within a trajectory's evolution. Therefore, robust cohesion is reinforced by having an attacking  pursuer randomly play against an evading flock in every iteration of the game as seen in \autoref{fig:robust_heading}. 


Since the orientations of neighboring agents are averaged with that of the singled-out bird, the information is inevitably propagated across the entire flock, and hence the murmuration.
 We found a Courant-Friedrichs-Lax (CFL) factor of $0.65$ to be a good hyperparameter during the integration of the HJI PDE\eqref{eq:HamiltonianOverall}. We adopt a warm-starting optimization scheme for the computation of the BRAT of each flock in our implementation so that we can leverage information accumulated during previous iterations when an optimization takes too long. We simply reload a BRAT from hard-disk from a different time step in the optimization process, and advance the Lax-Friedrichs integration scheme backward in time\footnote{An easier way to proceed backwards in time is to negate the dynamic equations in \eqref{eq:DubinsJadbabaie} and \eqref{eq:DubinsRelative}}.