\section{Materials and Methods.}
\label{sec:methods}
%
We cast the heading verification problem within a backward reachability analysis, where each agent within a flock have their respective payoff. Agents within a flock maintain cohesion by keeping track of their nearest neighbors' topological distance~\cite{Ballerini1232}. Capture occurs when an agent $j$ is on or within a circular bound (specified by a radius, $r$ in topological coordinates) around an agent $i$.  To avoid capture, flocks can dynamically split their value functions. The value functions can be stitched together under a smoothness of boundaries assumption (as is observed in natural swarms) in order to fend off attacks from a pursuing player as well as manage the computation of a large RCBRT verification problem.

%
 We synthesize the kinematics of agents that interact on distinct separate manifolds of the state space. These manifolds are called flocks in our description\footnote{Let the reader understand that we use the concept of a flock loosely. The value function could represent the cost of any large-scale control problem.}. For each flock, there corresponds local value functions that encode a desirable anisotropic density and structural patterns we would like to emerge.  These local value functions are constructed by  taking into cognizance each agent's topological interaction rule with its neighbors~\cite{Ballerini1232}. By solving for these respective value functions simultaneously and in parallel, we can stitch them together at the end of each integration step by considering the nature of the \textit{surfaces}\footnote{Surfaces can be  singular, dispersal, or universal in nature~\cite{Isaacs1965}.} of the value functions at the end of each integration step. These simple tricks allows us to compute large backward reach-avoid tubes that have eluded other scalability methods that have been so far introduced~\cite{Bansal, SylviaScalability, DecompChenHerbert}.
 
 \textbf{Assumptions}:
The many interacting subsystems under consideration employ
 %
 \begin{inparaenum}[(i)]
 	\item natural units of measurements that are the same for all agents; 
 	%
 	\item kinematics with linear speeds but with a capacity for orientation changes;
 	%
 	\item intra-flock agent interaction is restricted within unique and distinct state space manifolds; and by agents maneuvering their direction, a kinematic alignment is obtained;
 	%
 	\item inter-flock interaction occurs when a pursuer is within a threshold of capturing any agent within the murmuration.
 \end{inparaenum} 
%Biologists tell us that
%
Let us now formalize definitions that will aid the modularization of the problem into manageable forms.
%
\begin{definition}
	We define a \textit{flock}, $\flock$, of agents labeled $\{1, 2, \cdots, n\}$  as a collection of agents within a phase space $(\mc{X}, T)$ such that all agents within the flock interact with their nearest neighbors in a topological sense so as to preserve heading cohesion.
	\label{def:flock}
\end{definition}
%
%\begin{remark}
	Every agent within a flock has similar dynamics to that of its neighbor(s). Furthermore, agents travel at the same linear speed, $v$; the angular headings, $w$, however, may be different between agents. The state of a single agent $i$ within a flock will be defined as $\state^{(i)}$. Motion within a flock is described as a leaderless coordination of its $n$ autonomous agents.  Each agent's continuous-time dynamics evolves as
	%
	\begin{align}
		\dot{\state}^{(i)}(t) 
		= \begin{bmatrix}
			\dot{\state}^{(i)}_1(t) \\ \dot{\state}^{(i)}_2 (t) \\ \dot{\state}^{(i)}_3 (t)
		\end{bmatrix} = \begin{bmatrix}
			v(t) \cos \state_3^{(i)}(t) \\ v(t)\sin \state_3^{(i)}(t) \\ \langle w^{(i)}(t) \rangle_r
		\end{bmatrix}, \text{ where } 	\langle w^{(i)}(t) \rangle_r = \dfrac{w_i(t) + \sum_{j \in \mc{N}_i(t)} w_j(t)}{1+N_i(t)},
		\label{eq:DubinsJadbabaie}
	\end{align}
	%
	\text{ for agents } $i = \{1, 2, 3, ..., n\}$, where $t$ is the continuous-time index, $\state$ denotes points in $\ren$, and $w^{(i)}(t) \rangle_r$ is the average orientation of agent $i$ w.r.t its neighbors is $\langle w^{(i)}(t) \rangle_r$.

%Observe that the model of \eqref{eq:DubinsJadbabaie} is similar to the Dubins dynamics~\cite{Dubins1957} with the only difference being the update rule in the headings of the agents. This headings update rule is similar to the one used in statistical mechanics or  statistical physics (the physics of many interacting molecules)~\cite{Wilson1975}. We note here that it was recently adopted in control theory literature by Jadbabaie~\cite{JadbabaieCoord} in proving Viscek et al.'s self-driven particles simulations~\cite{Vicsek1995novel}.

\begin{definition}[Payoff of a Flock]
	To every agent $i$ within a flock, $F$, with finite number of agents $N$, we associate a payoff, $\valuefunc(\state^{(i)})$, that encodes the outcome of its topological interaction with its neighbors. 
	\label{def:payoff}
\end{definition}
%
%The target set of each local group of agents within a flock, we leverage Definition 5.7 of \cite{BasarBook}.
%
\begin{definition}[Definition 5.7,~\cite{BasarBook}]
	We consider the heading goal for a local group $F_j$\footnote{The subscript $j$ of $F$ is used to identify a distinct flock of agents within a murmuration on a state space.} within a murmuration of flocks $\flock$ as  an $N$-person differential game with a target set $\targetset_0^{j}$. The strategy $N$-tuple is said to be playable at $(\state_0, t_0)$ if it generates a trajectory $\traj(\cdot)$ such that $(\state(t), t) \in \targetset_0^{j}$ for finite $t$. Such a trajectory $\traj(\cdot)$ is said to be terminating.
\end{definition}
%
%as an $N-tuple$ of  we now use \autoref{def:payoff} to show that the collective behavior of a flock of agents is the union of the respective payoffs for each agent within the flock.
%%
%\begin{theorem}
%	Something sm,ething
%\end{theorem}

As mentioned in the introduction, we cannot solve the initial value problem in a classical way on $\ren \times [T, 0]$ in general. Viscosity solutions provide a particular means of finding a unique solution with a clear interpretation in terms of the generalized optimal control problem, even in the presence of stochastic perturbations. 
%
%Collective natural behavior in animals suggests that topological distance is used in maintaining group structure when density varies, so that the relationship between agents in a flock is not determined by the metric distance between nearest neighbors but rather the number of intermediate agents that separates two agent from one another.  Therefore, the neighbors of agent $i$ at time $t$ are those which are either within, or on a circle specified by a topological  distance, $r_c$. The topological metric is given  by the label of an agent and it quantifies the number of intermediate agents that separate two agents. 
%
%This is consistent with collective animal behaviors where individuals' bookkeeping on their neighbors' positions helps maintain the strength of an interaction when density varies or when they need to rorientation is its control input, given by the average of its own orientation and that of its neighbors. Instead of metric distance interaction rules that make agents very vulnerable to predation~\cite{Ballerini1232}, we resort to a topological interaction rule\footnote{With metric distance rules, we will have to formulate the breaking apart of value functions that encode a consensus heading problem in order to resolve the extrema of multiple payoffs; which is typically what we want to mitigate against during real-world autonomous tasks.}. 

Each agent within a flock interacts with a fixed number of neighbors, $n_c$, within a fixed topological range, $r_c$. This topological range %is given by the difference in the numerical label of individuals \cf \autoref{def:flock}, and 
is consistent with findings in collective swarm behaviors and it reinforces \textit{group cohesion}~\cite{Ballerini1232}. However, we are interested in \textit{robust group cohesion} in reachability analysis. Therefore, we let a pursuer, $\pursuer$, with a worst-possible disturbance attack the flock, and we take it that flocks of agents constitute an evading player, $\evader$. 

%
Returning to \eqref{eq:DubinsJadbabaie}, for a single flock, we now provide a sketch for the HJI formulation for a heading consensus problem. 

\subsection{Framework for Separated Payoffs}
%
We now make the following abstractions to enable our problem formulation.
%
A murmuration's global heading is predetermined and each agent $i$ within each flock, $\flock_j,\, (j=1, \cdots, n)$ in the murmuration has a constant linear velocity, $v^i$. An agent's orientation is its control input, given by the average of its own orientation and that of its neighbors. Instead of metric distance interaction rules that make agents very vulnerable to predators~\cite{Ballerini1232}, we resort to a topological interaction rule\footnote{With metric distance rules, we will have to formulate the breaking apart of value functions that encode a consensus heading problem in order to resolve the extrema of multiple payoffs; which is typically what we want to mitigate against during real-world autonomous tasks.}. 


\subsection{Nearest Neighbors Computation}
%
What constitutes an agent's neighbors are computed based on empirical findings and studies from the lateral vision of birds and fishes~\cite{Ballerini1232, JadbabaieCoord, Helbing20}. This lateral vision governs their anisotropic kinematic density and structure. Importantly, starlings' lateral visual axes and their lack of a rear sector reinforces their lack of nearest neighbors in the front-rear direction. As such, this enables them to maintain a tight density and robust heading during formation and flight.

\begin{definition}[Neighbors of an Agent]
	We define the neighbors of agent $i$ in flock $\flock_j$ at time $t$ as the set of all agents that lie within a predefined radius, $r_n$, of agent $i$. In every iteration of the game, we update an agent's neighbors as delineated in Algorithm \ref{alg:neighbors}.
\end{definition}
%


\begin{algorithm}[tb!]
\caption{Nearest Neighbors For Agents in a Flock.
		\label{alg:neighbors}}
\begin{algorithmic}[1]
	\State Given a set of agents $\bm{a} = \{a_1, a_2, \cdots, a_{n_a} \,| \,[a] = n_a\}$ 
	\Comment{$n_a$ agents in a flock  $\flock_k$.}
	\Function{UpdateNeighbor}{$n$}
	%\Comment{$N$: Total number of agents in flock.}
	%
	\For{$i$ in $1, \dots, n$}
		\Comment{Look to the right and update neighbors.}
		\label{alg:neighbors:line:lateral_vision_left}
		\For{$j$ in $i+1, \dots, n$}
			\State \textsc{Compare\_Neighbor($a[i]$, $a[j]$)}
		\EndFor
		%
		\For{$j$ in $i-1$ down to $0$}		
		\label{alg:neighbors:line:lateral_vision_right}
		\Comment{Look to the left and update neighbors.}
		\State \textsc{Compare\_Neighbor($a[i]$, $a[j]$)}
		\EndFor
	\EndFor
	%
	\For{each $a_i \in \flock_k, \, i=1, \cdots n_a$}
	 \Comment{Recursively update agents' headings.}
	 \State Update headings according to \eqref{eq:DubinsJadbabaie}.
	\EndFor 
\EndFunction
\end{algorithmic}
%
\hrule
%
\begin{algorithmic}[1]
	\Function{Compare\_Neighbor}{$a_1$, $a_2$}
	\Comment{$(a_1, a_2)$: distinct instances of AGENT.}
	\If{$|a_1.$label - $a_2.$label$|$ $< a_1.r^1_c$
		\Comment{$r^n_c$: agent $n$'s capture radius.}
		\State $a_1.$\textsf{\textsc{update\_neighbors}}($a_2$)}
	\EndIf
	\EndFunction
\end{algorithmic}
%
\hrule
%
\begin{algorithmic}[1]
	\Procedure{Agent}{$a_i$, \textsf{Neighbors}=$\{\}$}  
	\Comment{Neighbors: Set of neighbors of this agent.}
	\State \Comment{Agent $a_i$ with attributes \textsf{label $\in \bb{N}$, avoid and capture radii, $r_a, r_c$.}}
	\Function{\textsf{\textsc{update\_neighbors}}}{\textsf{neigh}}	
	\If{\textsf{length(neigh)}$ > 1$}
		\Comment{Multiple neighbors.}
		\For{each \textsf{neighbor} of \textsf{neigh}}
			\State \textsc{\textsf{update\_neighbors}}(\textup{neighbor})
			\Comment{Recursive updates.}
		\EndFor 
	\EndIf
	%
	\State Add \textup{\textsf{neigh}} to \textsf{Neighbors}
	\EndFunction
	\EndProcedure
\end{algorithmic}
\end{algorithm}
%

The algorithm for computing the nearest neighbors is given in Algorithm \autoref{alg:neighbors}. On lines \autoref{alg:neighbors:line:lateral_vision_left} and \autoref{alg:neighbors:line:lateral_vision_right} of Algorithm \autoref{alg:neighbors}, cohesion is reinforced by leveraging the observations above. 
Each agent within a flock $\flock_j$ interacts with a fixed number of neighbors, $n_c$, within a fixed topological range, $r_c$. This topological range %is given by the difference in the numerical label of individuals \cf \autoref{def:flock}, and 
is consistent with findings in collective swarm behaviors and it reinforces \textit{group cohesion}~\cite{Ballerini1232}. However, we are interested in \textit{robust group cohesion} in reachability analysis. Therefore, we let a pursuer, $\pursuer$, with a worst-possible disturbance attack the flock, and we take it that flocks of agents constitute an evading player, $\evader$. 

\subsection{Global Isotropy via Local Anisotropy}
%
It has been observed that structural anisotropy is not merely an effect of a preferential velocity in animal flocking kinematics but rather an explicit effect of the anisotropic interaction character itself.
 
\textit{ A distinct idea is that the mutual position chosen by the animals is the one that maximizes the sensitivity to changes of heading and speed of their neighbors (9). According to this hypothesis, even though vision is the main mechanism of interaction, optimization determines the anisotropy of neighbors, and not the eye's structure. There is also the possibility that each individual keeps the front neighbor at larger distances to avoid collisions. This collision avoidance mechanism is vision-based but not related to the eye's structure.
}%

Because of the robust group cohesion philosophy, we take it that every agent within a flock $flock_j$, under attack is in relative coordinates with a pursuer, $\pursuer^j$. By averaging the heading of individual agents orientations with that of its neighbors\cf \eqref{eq:DubinsJadbabaie}, individual agents within a flock can achieve  instantaneous  response to danger when a pursuer is nearby. \todo{In this specialized case,  capture does not necessarily occur, because the $\evader$ and $\pursuer$'s speeds and maximum turn radius are equal}: if both players start the game with the same initial velocity and orientation, the relative equations of motion show that $\evader$ can mimic $\pursuer$'s strategy by forever keeping the starting radial separation~\cite{Merz1972}. As such, the \textit{barrier} is closed and the \textit{game of kind} is to determine the surface. However, owing to the high-dimensionality of the state space, we cannot resolve this barrier analytically, hence we resort to numerical approximation methods -- in particular, we leverage a parallel Lax-Friedrichs integration scheme~\cite{Crandall1984} which we implement in Cupy~\cite{CuPy} in order to provide a \textit{consistent} and \textit{monotone} solution to the Hamiltonians of these HJI equations\footnote{Consistent solutions to HJ equations are those whose explicit marching schemes via discrete  approximations to the HJ IVP agree with the nonlinear HJ solution~\cite{Crandall1984Approx}. Such schemes are said to be \textit{monotone} \eg on $\left[-\reline, \reline\right]$ if the numerical approximation to the vector field of interest is a nondecreasing function of each argument of the discrete approximation to the vector field.}. 
%
\begin{figure}[tb!]
\centering
\includegraphics[width=\columnwidth]{figures/flock_pursuer.jpg}
\caption{Illustration of robust heading consensus for a flock. Each agent within the Evading player's control are in relative coordinates w.r.t a pursuing adversary. Each agent (identified by index $i$) within the evading flock is parameterized by three state components: the linear velocities $(\state_1^{(i)}, \state_2^{(i)})$, and heading $w^{(i)}$. When we need to distinguish an agent within a flock from another flock, we shall use the index of the flock \eg $j$ as a subscript for a particular \eg  $\state_1^{(i)_j}$.}
\label{fig:robust_heading}
\end{figure}

Therefore, for an agent $i$ within a flock with index $j$ in a murmuration, the equations of motion under attack from a predator (see \autoref{fig:robust_heading}) is %and terminal conditions are given by
%
\begin{align}
\left[\begin{array}{c}
\dot{\state}_1^{(i)_j}(t) \\ \dot{\state}_2^{(i)_j} (t) \\ \dot{\state}_3^{(i)_j} (t)
\end{array}\right] = \begin{bmatrix}
-v_e^{(i)_j}(t) + v_p^{(j)} \cos \state_3^{(i)_j}(t) + \langle w_e^{(i)_j} \rangle_r \state_2^{(i)_j}(t)
%
\\
v_p^{(i)_j}(t)\sin \state_3^{(i)_j}(t) - \langle w_e^{(i)_j} \rangle_r \state_1^{(i)_j}(t)
\\ 
w_p^{(j)}(t) - \langle w_e^{(i)_j}(t) \rangle_r
\end{bmatrix} \, \text{for } i=1,\cdots, N
\end{align}
%
where $N$ is the number of flocks. Read ${\state}_1^{(i)_j}(t)$: the first component of the state of an agent $i$ at time $t$ which belongs to the flock $j$ in the murmuration. As mentioned in Section \ref{sec:notations}, the evading player at anytime has controls $\{\control^1, \control^2, \cdots \control^n\}$ for agents $i=1, \cdots, n$ completely  under its will. Solving for such complex backward reach-avoid tube is akin to splitting the state space into a number of parts separated by surfaces. 
We assume that the \textit{value} of a flock heading control (differential game) exists, so that the terminal value function is
%
\begin{subequations}
\begin{align}
&\frac{\partial \lowervalue_j}{\partial t} + \lowerham_j (t; \state, \bm{u}, \bm{v}, \lowervalue_{\state_j}) = 0, \,\, t\in \left[0, T\right]\, \state \in \ren  \\
&\lowervalue_j(\state, T) = g_j(x(T)), \quad \state \in \reline^m , \quad j = 1, \cdots, n
\label{eq:lower_visc_boundary_surfaces}
\end{align}
\label{eq:lower_visc_surfaces}
\end{subequations}
%
with lower Hamiltonian, 
%
\begin{align}
&\lowerham_j (t; \state, \bm{u}, \bm{v}, p) = \max_{u \in \mathcal{U}} \min_{v \in \mathcal{V}} \, \langle f_j(t; \state, \bm{u}, \bm{v}), p_j  \rangle, \quad j = 1, \cdots, n,
\label{eq:lower_visc_ham_surfaces}
\end{align}
%
where $n$ is the total number of separate and distinct flocks on a state space. We follow ~\cite{Isaacs1965}'s definition by ascribing to these surfaces the term \textit{singular surfaces}, which signifies $(n-1)$-manifolds in $n$-space.  On the respective surfaces, we are concerned with the resolution of the respective \textit{values} \ie  $\{\valuefunc_1, \cdots, \valuefunc_n\}$  for distinct flocks $\{\flock_1, \flock_2, \cdots, \flock_n\}$ on the distinct surfaces of the state spaces $\{\stspace_1, \stspace_2, \cdots, \stspace_n\}$. The reader should note that $\stspace = \stspace_1 \cup \stspace_2 \cup \cdots \cup \stspace_n$. %On each surface, the solution to the respective $\valuefunc_i$ is $C_1$ smooth. 
Similarly, we  attribute the term \textit{in the small} to determine the smooth parts of the singular surface solution, and when they are stitched together into the total solution, we shall describe them as \textit{in the large}. 


%W this would introduce computational complexity issues and resolving the respective boundaries of the value functions.
To encourage robust cohesion, a pursuer can attack any flock or a group of flocks within the murmuration from two distinct surfaces, which we call a $\pursuer$ or an $\evader$ direction. We call the side of the surface reached after penetration in the $\pursuer-[\evader-]$ direction the $\pursuer-[\evader]$ \textit{side}. %Points $\state$ on the so oriented surface 
There exists at least one value $\bar{\alpha}$ of $\alpha$ such that if $\alpha = \bar{\alpha}$, no vector in the $\beta$-vectogram\footnote{A $\beta-$vectogram is the resulting state space when a the strategy $\beta$ is applied in computing the optimal control law for an agent.} penetrates the surface in the $E$-direction. Similar arguments can be made for $\bar{\beta}$ which prevents penetration in the $P$-direction. We adopt ~\cite{Isaacs1965}'s terminology and call these surfaces semi-permeable surfaces (SPS).

We are dealing with a family of games based on different starting points for local flocks that on the whole constitute a murmuration; therefore, we construct the respective target sets for each individual agent  implicitly on the state space as $\valuefunc: [-T, 0]\times \mc{X} \rightarrow \bb{R}$. In starlings, agents move in local flocks of six to seven nearest neighbors~\cite{Ballerini1232} in order to preserve cohesion and heading consensus. Therefore, we define the target set and the tube as
%
\begin{align}
\targetset_0 &= \left\{ \state \in \bar{\openset} | \valuefunc(\state, 0) \le 0 \right\}, \\
\mathcal{L}([\tau, 0],  \mathcal{L}_0) &= \left\{\state\in \bar{\openset}  | \valuefunc(\state, \tau) \le 0\right\}
\end{align}
%
where $\tau \in  [-T, 0]$. %Abusing notation and dropping the $i$th superscript for an agent, we construct $\valuefunc(\cdot)$ as 
%
%\begin{align}
%	\valuefunc(\state, 0) = \sqrt{\state_1^2 + \state_2^2} - n_c
%\end{align}
%%
%where $n_c$ is the capture radius, equivalent to the topological range for a flock as reported in~\cite{Ballerini1232}. 

Throughout the game, we assume that the roles of $\pursuer$  and $\evader$ do not change, so that when capture can occur, a necessary condition to be satisfied by the saddle-point controls of the players is the Hamiltonian. For  a flock, $j$, this is the total energy exerted by each agent $i$ in the flock so that we can write the total Hamiltonian of a murmuration as 
%
\begin{align}
\bm{H}(\state, p) = \sum_{j=1}^{N_f} \sum_{i=1}^{N_a} H^{(i)}_j(\state, p)
\label{eq:HamiltonianOverall}
\end{align}
%
where $N_f$ is the total number of distinct flocks in a murmuration, $N_a$ is the total number of agents within flock $j$. $H^{(i)}_j(\state, p)$ is the Hamiltonian of flock $j$,  given by
%
\begin{align}
H^{(i)}_j(\state, p)&= - \left(\max_{w_e^{(i)_j} \in [\underline{w}_e^j, \bar{w}_e^j]}  \min_{w_p^{(i)_j}  \in [\underline{w}_p^{j}, \bar{w}_p^j]}  \left[\begin{array}{ccc} p_1^{(i)_j}(t) & p_2^{(i)_j}(t) & p_3^{(i)_j}(t) \end{array}\right] \right. \nonumber \\ 
%
& \qquad\qquad\qquad \left. 
\begin{bmatrix}
-v_e^{(i)_j}(t) + v_p^{(j)} \cos \state_3^{(i)_j}(t) + \langle w_e^{(i)_j} \rangle_r (t) \state_2^{(i)_j}(t)
\\ 
v_p^{j}(t)\sin \state_3^{(i)_j}(t) -\langle w_e^{(i)_j} \rangle_r (t) \state_1^{(i)_j}(t)
\\ 
w_p^j(t) - \langle w_e^{(i)_j}(t) \rangle_r
\end{bmatrix}\right),
\label{eq:Hamiltonian}
\end{align}
%
where $w_e^{(i)_j}$ is the heading of an evader $i$ within a flock $j$ and $w_p^{(j)}$ is the heading of a pursuer aimed at flock $j$; $\underline{w}_e^{(i)_j}$ is the orientation that corresponds to  the orientation of the agent with minimum turn radius among all the neighbors of agent $i$, inclusive of agent $i$ at time $t$; similarly, $\bar{w}_e^{(i)_j}$ is  the maximum orientation among all of the orientation of agent $i$'s neighbors. For the pursuer, its minimum and maximum turn rates are fixed so that we have $\underline{w}_p^{j}$ as the minimum turn bound of the pursuing vehicle, and $\bar{w}_p^j$ is the maximum turn bound of the pursuing vehicle. Henceforth, we drop the templated time arguments for ease of readability. Rewriting \eqref{eq:Hamiltonian}, we find $\bm{H}^{(i)}_j(\state, p)$ as
%
\begin{align}
\begin{split}
&\bm{H}^{(i)}_j(\state, p)=- \left(\max_{w_e^{(i)_j} \in [\underline{w}_e^j, \bar{w}_e^j]}  \min_{w_p^{(i)_j}  \in [\underline{w}_p^{j}, \bar{w}_p^j]}  
\left[
-p_1^{(i)_j} v_e^{(i)_j} + p_1^{(i)_j} v_p^{j} \cos \state_3^{(i)_j} 
\right. \right.  \\ 
%
& \left. \left. +  p_1^{(i)_j}\langle w_e^{(i)_j} \rangle_r \state_2^{(i)_j} 
%
+ p_2^{(i)_j} v_p^{j} \sin \state_3^{(i)_j} - p_2^{(i)_j} \langle w_e^{(i)_j} \rangle_r  \state_1^{(i)_j} 
%
+ p_3^{(i)_j}\left(w_p^j - \langle w_e^{(i)} \rangle_r\right)
\right] 
%
\right),
\\
%
&=p_1^{(i)_j} \left(v_e^{(i)_j} - v_p^{j} \cos \state_3^{(i)_j}\right) -  p_2^{(i)_j} v_p^{j} \sin \state_3^{(i)_j}  \\
%
& \qquad + \left(		
\max_{w_e^{(i)_j} \in [\underline{w}_e^j, \bar{w}_e^j]}  \min_{w_p^{(i)_j}  \in [\underline{w}_p^{j}, \bar{w}_p^j]} \begin{bmatrix}
\langle w_e^{(i)_j}\rangle_r \left(p_2^{(i)_j} \state_1^{(i)_j} - p_1^{(i)_j}  \state_2^{(i)_j} \right) 
% 
\\ 
- p_3^{(i)_j} \left(w_p^j - \langle w_e^{(i)}\rangle_r\right)
\end{bmatrix}
\right).
\end{split}
%
\label{eq:ham_flock_interm}
\end{align}
%
In this work, we consider the special case where the linear speeds of the evaders and pursuer are equal \ie $v_e = v_p = +1$.  Then the Hamiltonian $\bm{H}^{(i)}_j(\state, p)$ in \eqref{eq:ham_flock_interm} becomes 
%
\begin{align}
\bm{H}^{(i)}_j(\state, p) &= p_1^{(i)_j} \left(1 - \cos \state_3^{(i)_j}\right) - p_2^{(i)_j} \sin \state_3^{(i)_j} - \underline{w}_p^j \bigg|p_3^{(i)_j}\bigg| \nonumber 	\\
& \qquad\qquad  
 + \bar{w}_e^j \bigg|\left(p_2^{(i)_j} \state_1^{(i)_j} - p_1^{(i)_j}\state_2^{(i)_j} + p_3^{(i)_j}\right)\bigg|  .
\end{align}
%
From \eqref{eq:HamiltonianOverall}, we thus write
%
\begin{align}
\bm{H}(\state, p) &= \sum_{j=1}^{N_f} \sum_{i=1}^{N_a} p_1^{(i)} \left(1 - \cos \state_3^{(i)_j}\right) - p_2^{(i)_j} \sin \state_3^{(i)_j} - \underline{w}_p^j \bigg|p_3^{(i)_j}\bigg|  \nonumber 	\\
& \qquad\qquad %+\bar{w}_e^j |p_3^{(i)_j}| 
+ \bar{w}_e^j \bigg| p_2^{(i)_j} \state_1^{(i)_j} - p_1^{(i)_j}\state_2^{(i)_j} + p_3^{(i)_j}\bigg|.
\label{eq:HamiltonianCompletada}
\end{align}
%