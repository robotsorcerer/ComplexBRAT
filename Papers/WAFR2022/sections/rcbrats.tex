\section{Flocks' Robustly Controllable BRATs}
\label{sec:rcbrats}

\begin{figure}[tb!]
	\centering
	\begin{tabular}{ccc} 
		\includegraphics[height=12em,width=10em]{figures/flock_0_init.jpg} 
		&
		\includegraphics[height=12em,width=10em]{figures/flock_1_init.jpg} 
		& 		
		\includegraphics[height=12em,width=10em]{figures/flock_2_init.jpg} 
		\\
		\includegraphics[height=12em,width=10em]{figures/flock_0_final.jpg} 
		&
		\includegraphics[height=12em,width=10em]{figures/flock_1_final.jpg} 
		& 		
		\includegraphics[height=12em,width=10em]{figures/flock_2_final.jpg} 		\\
%	\end{tabular}
%	%
%	%
%	%midrule 
%	%
%	\centering
%	\begin{tabular}{ccc} 
		\includegraphics[height=12em,width=10em]{figures/flock_3_init.jpg} 
		&
		\includegraphics[height=12em,width=10em]{figures/flock_4_init.jpg} 
		& 		
		\includegraphics[height=12em,width=10em]{figures/flock_5_init.jpg} 
		\\
		\includegraphics[height=12em,width=10em]{figures/flock_3_final.jpg} 
		&
		\includegraphics[height=12em,width=10em]{figures/flock_4_final.jpg} 
		& 		
		\includegraphics[height=12em,width=10em]{figures/flock_5_final.jpg}	
	\end{tabular}
\caption{\footnotesize Top Rows: Initial zero-level set for flocks initialized from different initial conditions.  Bottom: Interface of the respective evading flock under attack from a pursuer after specific Lax-Friedrichs' integration.   (Metric reach radius=$0.2m$, Avoid Radius=$0.2m$).} 
\label{fig:rcbrats}
\end{figure}

Note that the symmetry between non-consecutive flock labels \eg flock 1 and flock 3's RCBRAT is because the we multiplied the initial position of a flock's state by $-1$.