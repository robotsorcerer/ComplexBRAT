\section{Flight Model and Control}
\label{sec:light_mpc}

We consider the classic dart-shaped paper airplane for our numerical analysis and experiment. The paper plane is characterized by a point mass whose longitudinal motions are described by four differential equations. The integral of these differential equations yields three longitudinal paths \ie the 
%
\begin{inparaenum}[(1)]
	\item constant-angle descent; 
	%
	\item vertical oscillation; and
	%
	\item loop.
\end{inparaenum} 
%
The paper plane is assumed made out of plain paper, with a weight of $0.003kg$, wing span of $12cm$ and a length of $28cm$. We fix the angle of attack at $9.3^\circ$, producing the best configuration lift-drag ratio of $5.2$. The chosen angle of attack produces the ~\cite{FlightDynamicsStengel}. 

\todo{
\begin{itemize}
	\item Describe rationale for using drones;
	%
	\item Describe drone dynamics;
	%
	\item Describe MPC controller;
	%
	\item Add result about MPC and indicate why a basic MPC scheme fails to obey state convergence constraints;
	%
	\item Add result about why MPC fails to obey input constraints.
\end{itemize}
}