
%%%%%%%%%%%%%%%%%%%%%%%%%%%%%%%%%%%%%%%%%%%%%%%%%%%%%%%%%%%%%%%%%%%%%%%%%%%%%%%%
\section{INTRODUCTION}

In this paper, we shall concentrate on finding optimal policies in a large scale system that satisfy specified constraints in the inputs and states of the dynamical system. We term such constraints as ``safety-preserving". These constraints usually involve states and input (or actuators) constraints that must be satisfied in a least restrictive sense.  

Safety-preserving problems usually arise in the interaction among multiple agents that share a physical  space such as cyber-physical systems (CPS). CPS examples  may include modern manufacturing assembly lines where humans and machines jointly work together for   products delivery to a supply chain being controlled by computer software resources, personalized interoperable medical devices, autonomous cars on a highway, (almost unmanned) long-hauled passenger flights, or multidimensional allocation processes with multiple constraints such as cargo loading problems (with weight and size restrictions)~\cite[Ch. II]{AppliedDPBellman}.

The ``physical" and ``cyber" couplings of such systems is critical in modern CPS infrastructure: finding feasible control laws  in the presence of complex dynamics; solving for collision avoidance schemes in real-time  for complex multi-agent systems, or navigating in uneven terrains while maintaining stability -- all require a systematic orchestration of local control laws that must be carefully planned and executed. Therefore,  the safety analysis of combined CPS systems in the presence of sensing, control, and learning becomes timely and crucial. 

%Differential optimal control theory and games offer a systematic paradigm for resolving the safety of multiple agents interacting over a shared space. Both problems resolve a (robustly) optimal controller by analyzing the Hamilton-Jacobi Bellman (HJB) or its Isaacs (HJI) counterpart.  As HJ-type equations have no classical solution for almost all \textit{practical} problems, stable numerical and computational methods need to be brought to bear in order to produce solutions with (approximately) optimal guarantees. 

%With essentially non-oscillating (ENO)~\cite{OsherShuENO} Lax-Friedrichs~\cite{CrandallLaxFriedrichs} schemes, consistent and monotone solutions to HJ Hamiltonians can be resolved via explicit marching schemes that solve for unique (viscosity) solutions~\cite{Crandall1983viscosity, Crandall1984} to HJ-type equations with high accuracy and precision \textit{on a grid}~\cite{Crandall1984Approx, Sethian87Numerical, SethianLSBook},  unique (viscosity) solutions to HJ-type equations with high accuracy and precision \textit{on a mesh}. Employing meshes for resolving inviscid Euler equations whose solutions are the derivatives of HJ equations, these methods scale exponentially with state dimensions, making them ineffective for complex systems  -- a direct consequence of \textit{curse of dimensionality}~\cite{Bellman1957}. %Truncated power series methods~\cite{Jacobson1968new, JacobsonMayne, 	DenhamDDP, TodorovCDC} are successive approximations of HJ value functions; however, these limit the stability region of the resulting approximate controller, and require a careful tuning of the approximate controller such that it has a direct effect on the original optimal control problem.  In addition, stability is not easily guaranteed for series approximated HJ value functions where it is generally assumed that the highest-ordered terms in the series truncation dominate neglected higher-order terms.

