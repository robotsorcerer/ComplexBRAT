\section{Related Work}

Robust Controlled Backward reachable tubes (RCBRTs)~\cite{Mitchell2021, Mitchell2005} are a class of reachability analysis tools, where the basic version of the problem involves an adversarial game setting for two dynamical systems, typically possessing Dubins car vehicular dynamics~\cite{Dubins1957}, such that a \textit{pursuer}, operating under a nonantitipative strategy~\cite{BasarBook}, aims to drive an \textit{evader} toward a \textit{target set} whilst avoiding an unsafe set of states under the worst-possible disturbance at all times; this nonanticipative strategy lends usefulness to general dynamical systems. Empirical results exist that demonstrates the feasibility of learning the optimal value function for up to ten dimensions~\cite{Bansal} and generating robustly optimal trajectories. However, these have the drawback of not being amenable to sampling-based strategies owing to their optimal construction: from a single point in the state space, distinct optimal trajectories may arise that lead to a single point in the target space~\cite{Mitchell2020}. And a dense sampling of the competing inputs is non-tenable either owing to computational sampling complexity. Also, a fundamental assumption in these methods is that the dynamics of the respective vehicles are distinctly separable, and discontinuous at so-called cross-over points on the surface of the RCBRT; this is not always the case in many physical systems, where there exists coupled states among interacting agents. Our work avoids this dense sampling computational conundrum by using a KL-E$^3$ ergodic measure for exploration in providing an information-rich dynamics search.

Other approaches leverage minimizing an ergodic metric between the time-averaged statistics of a robot and the spatial distribution of a target search domain such that an agent can sample measurements in high utility regions defined by the spatial distribution.


\textbf{Bayesian Optimization}:

\textbf{Active Exploration}:

\textbf{Variable Resolution State Space Partitioning}:

\textbf{Ergodicity and Ergodic Exploration}