\section{INTRODUCTION.}
%
\noindent Designed cyberphysical systems (CPS) are  a complex interconnection of control systems, sensors, and their software whose communication protocols have created complex entanglements with interactions that are difficult to analyze.  CPS are traditionally engineered to sense and interact with the physical world ``smartly". Modern cyberphysical systems may include modern manufacturing assembly lines where humans and machines jointly work to deliver products to a supply chain controlled by computer software resources, personalized interoperable medical devices, autonomous cars on a highway, (almost unmanned) long-hauled passenger flights, or general logistics inter alia. 

The ``physical" and ``cyber" couplings of such systems is critical in modern CPS infrastructure: generating control laws -- where dynamics may be complex; planning and executing in real-time collision avoidance schemes in uneven terrains, or sensing efficiently in the presence of multiple agents -- all require deep integration and the actions of system components must be planned meticulously. Therefore,  the safety analysis of combined CPS systems in the presence of sensing, control, and learning becomes timely and crucial. Differential optimal control theory and games offer a powerful paradigm for resolving the safety of multiple agents interacting over a shared space. Both problems rely on a resolution of the Hamilton-Jacobi-Bellman (HJB) or the Hamilton-Jacobi-Isaacs (HJI) equation in order to solve the control problem.  As HJ-type equations have no classical solution for almost all \textit{practical} problems, stable numerical and computational methods need to be brought to bear in order to produce solutions with (approximately) optimal guarantees. 

With essentially non-oscillating (ENO)~\cite{OsherShuENO} Lax-Friedrichs~\cite{CrandallLaxFriedrichs} schemes applied to numerically resolve HJ Hamiltonians~\cite{Evans1984}, we can now obtain  unique (viscosity) solutions to HJ-type equations with high accuracy and precision \textit{on a mesh}. Employing meshes for resolving inviscid Euler equations whose solutions are the derivatives of HJ equations, these methods scale exponentially with state dimensions, making them ineffective for complex systems  -- a direct consequence of \textit{curse of dimensionality}~\cite{Bellman1957}. Truncated power series methods~\cite{Jacobson1968new, JacobsonMayne, 	DenhamDDP, TodorovCDC} are successive approximations of HJ value functions; however, these limit the stability region of the resulting approximate controller, and require a careful tuning of the approximate controller such that it has a direct effect on the original optimal control problem.  In addition, stability is not easily guaranteed for series approximated HJ value functions where it is generally assumed that the highest-ordered terms in the series truncation dominate neglected higher-order terms.

Therefore in subject matter and emphasis, this paper reflects the influences described in the foregoing. As a result, we focus on computational techniques because almost all \textit{practical} problems cannot be analytically resolved. To analyze safety, we cast our problem formulation within the framework of \textit{Cauchy-type} HJ equations~\cite{Crandall1983viscosity}, and we specifically resolve the scalable safety problem by solving the terminal value in the HJ PDE within the framework of Mitchell's \textit{robustly controlled backward reachable tubes}~\cite{Mitchell2020}. %Extensions to \textit{Dirichlet-type} HJ equations are straightforward.  

In this sentiment, new computational techniques are introduced including
%
\begin{inparaenum}[(i)]
	\item iterative Galerkin approximation of large value functions;
	%
	\item finite difference approximation schemes with error estimates (essentially, an extension of ~\cite{Crandall1984} on reduced Hilbertian spaces); and
	%
	\item  analytic saddle solutions to approximated HJ value functions.
\end{inparaenum} 
%
to synthesize approximately optimal control laws (essentially, saddle-point solutions) \todo{with stability guarantees} for resolving the terminal value in the viscosity solutions to HJI value functions.


%Owing to the broad range of applications from vertical farming, distributed systems, and  biology, questions of reachability, viability, and invariance have received broad attention in the dynamics and control literature lately. For continuous and hybrid systems, the scalability of  verification and validation methods is timely for analyzing the safety of complex infrastructure such as ground traffic management systems~\cite{GroundTrafficMgt}, air traffic management systems~\cite{Tomlin2000Game}, flight control systems~\cite{Mitchell2020, SylviaScalability}, the invariance of quadrupedal locomotion~\cite{QuadCBF}, or adaptive cruise control~\cite{CBFCruiseControl} inter alia. 
%
%Essential to these systems are the following,
%%
%\begin{inparaenum}[(i)]
%	%\item an agent must become ``environmentally aware": that is, it must have an understanding of the interplay between the (sometimes non-stationary) environmental \textit{dynamics}, constraints (e.g. collision avoidance in high dimensional robot manipulation tasks),
%	%
%	\item an agent must efficiently use its resources to discover a rich-enough dynamics of its environment (ideally in real-time) -- its current state -- in order to do effective planning and control;
%	%
%	\item having tallied up its planning strategy, robust trajectories must be executed within a safety envelope such that an agent's execution of its control laws does not hamper the control strategy of other interconnected agents.
%\end{inparaenum}   
%
%\begin{comment}
%An agent, therefore, must understand the dynamics of its environment \textit{within a reasonable amount of time}; then the agent's confidence in its understanding of the environment being reasonably high, it may \textit{execute an (ideally optimal) control policy} (or sequence of control policies)  on a given task, under a safety-critical objective. More often than not, we would want the agent to execute this policy \textit{robustly} (e.g. factoring safety of expensive hardware parts on a state-of-the-art robot whilst executing a trajectory in a cluttered environment).
%
%The foregoing problem necessitates our rethinking of the solution of traditional systems identification and control, as well as learning problems. Analytical modeling methods will often not suffice in this setup. In sensitive applications, machine learning methods must learn to balance the cost of exploration versus that of exploitation \textit{almost in real-time} before a controller kicks in. While reinforcement learning~\cite{sutton2018reinforcement} may seem attractive for these sort of problems, its promise for simultaneously  learning the dynamics and \textit{robust} policies in an (almost) real-time  time-efficient manner remains elusive.  Non-parametric approaches such as deep reinforcement learning~\cite{lillicrap2015continuous, mnih2013playing}, while holding a great promise for solving control and planning problems in complex spaces, often suffer from sensitivity of the parameter space, as well as lack of robustness and safety on real-world autonomous control tasks~\cite{Summers2017, ogunmolu2018minimax, AmodeiSafety}.
%
%For multiple input-multiple output complex systems, the dynamics may be coupled along certain dimensions, discrete, continuous, possess an uneven harmony of discrete and continuous dynamics, or with discontinuity in the solution space, so that traditional gradient-based methods become impractical as solution mechanisms.  As such, safety and secure autonomy in these systems requires a proper calibration and design in high dimensions. A useful technique to analyze these systems is to devise computational validation techniques in simulation. However, a simulation must be able to reason in high dimensions. While methods do exist to simulate robustly optimal systems, there are tools lacking that provide theoretical guarantees on optimality and robustness whilst demonstrating the feasibility of existence and uniqueness of solutions in reasonably high-dimensional systems.
%\end{comment}
% 
%\todo{In this work, we focus  on generalized solutions of Hamilton-Jacobi equation.  The major contribution of this article is that we prescribe an incremental spatial decomposition of high-dimensional systems that preserves the optimality guarantees of backward reachability problems up to a specified error bound}. %This scales linearly with the dimensionality of the problem at hand -- a major computational advantage in comparison with state-of-the-art methods in backward reachability analysis that scale exponentially in time-space. For a proof-of-concept, we extend recent results in the time-dependent HJ solutions to dynamic games in high-dimensional control problems by considering proper generalized decomposition  as sums of the products of tensors over the physical space at each time step. %: $u_t(x, y, z) = \sum_i F^i_{x_i} (x) \, F^i_{y_i} (y) \, F^i_{z_i} (z)$. %we sidestep methods that employ Nagumo's set invariance theory~\cite{Nagumo1942, blanchini1999set} since those employ ordinary differential equations with an underlying assumption of Lipschitz continuity in the state space and an assumption of known barrier certificates. Instead, 
% 
% \todo{This problem has important consequence not just in control reachability domains but in many aspects of optimal control and approximate optimal learning methods that use global approximations to the Hamilton-Jacobi-Bellman equation in resolving control laws or policy learning in large state spaces. In deep reinforcement learning for example, the value function is approximated in global space with no stability measures for the gradients as surrogate value functions for the Bellman DP function are optimized. Leveraging Lax-Friedrichs~\cite{LevelSetsBook} methods that scale in a controlled manner to large state spaces, and parallelizable in such a way that it takes into account the advanced computational resources available today is attractive because it opens the door for more robust policies for real-world agents. }


\begin{comment}
\noindent \textbf{Contributions}:

 Since large systems often have sparse dynamics, we use fast Krylov-subspace simulation methods based on the Arnoldi or Lanczos iterations. Our implementation produces accurate counter-examples when properties are violated and, in the extreme case with sufficient problem structure, is shown to analyze a system with up to $n=x$ states. The basic notion of our contribution applies a variable resolution partitioning of high-dimensional state spaces using an extension of Moore and Atkeson's Parti-Games algorithm~\cite{Moore1995}.  Within each partition, since dense state spaces are often sparse, we exploit the sparsity of the state space and let the partitioning function decide upon a coarse or fine partitioning of cells. %Within each cell that an agent may transition into under a Markov Decision Task construction, we pose the solution to the next step of the trajectory as viscosity solutions to Hamilton-Jacobi (HJ)-type  nonlinear dynamics of the \textit{Dirichlet problem}

\begin{itemize}
	\item We first provide an ergodic metric and optimization analysis similar to~\cite{Abraham2021} that allows an agent's time-averaged statistics over its states to match the spatial statistics of a defined target distribution that intersects those states.
	%
	\item We adaptively vary a cell's resolution (within a grid) based on a sparsity measure that determines if the information states in the grid are soarse or not;
	%
	\item We then solve the next step of the Markov Decision Task, using the viscosity solution to the HJI problem in computing an optimal controller under the worst possible disturbance for an agent.
\end{itemize}
\end{comment}

In order to analyze the safety of emerging CPS systems given the computational and memory drawbacks of level sets methods, it is the opinion of the authors that  
%
\begin{itemize}
	\item easily implementable approximation schemes with %closed-loop
	 stability guarantees; 
	%
	\item stability in well-defined regions of the state space where approximation is guaranteed to work; % (e.g. for finite series truncations); and
	%
	\item and   low run-time computation and memory requirements that address the jugular of the curse of dimensionality; whilst
	%
	\item providing bounds on the error in the approximation,
\end{itemize}   
%
are the best means for tackling this problem.

%Leveraging domain decomposition methods, we introduce modularity that considers local spatial nonlinearities and intefaces with model system reality. In this sentiment, we introduce a novel set of techniques including: 
%		\begin{inparaenum}[(i)]
%			\item sequentially truncated tensor representations of high degree of freedom value functions; 
%			%
%			\item iterative projections to reduced basis, based on sequentially truncated high-order orthogonal iterations;
%			%
%			\item providing tight bounds on  the approximate control law from the reduced basis to ensure an improvement in the original cost; and
%			%
%			\item a verification of our hypothesis and formulation on a high-dimensional safety verification problem for a team of pursuer-evader Dubins vehicles.
%		\end{inparaenum} 

The rest of this paper is organized as follows: we introduce common notations and definitions in \autoref{sec:notations}; \autoref{sec:related} describes the concepts and topics we will build upon in describing our proposal in \autoref{sec:methods}; we present results and insights from experiments in \autoref{sec:results}. We  conclude the paper in \autoref{sec:conclude}. This work is the first to systematically provide a rational incremental decomposition scheme that provides approximation guarantees on regions of the state space where approximate HJ control laws are valid as well as provide a rational analysis for high-dimensional verification of nonlinear systems with guarantees.  %\todo{Add more stuff that encompasses the problems we address in this paper here.} 