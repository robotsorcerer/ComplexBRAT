\section{Results and Discussion.}
\label{sec:results}
We now provide results and analysis of the proposed numerical algorithm on benchmark control problems.

\subsection{Time Optimal Control of the Double Integrator}
%
Here, we analyze our proposal on a time-optimal control problem. Specifically, we consider the  double integral plant which has the following second-order dynamics
%
\begin{align}
	\ddot{\state}(t) = \control(t).
	\label{eq:double_integ}
\end{align}
%
This admits bounded control signals $\mid \control(t) \mid \le 1$ for all $t$. After a change of variables,we have the following system of first-order differential equations
%
\begin{align}
	\dot{\state}_1(t) &= \state_2(t),\quad
	\dot{\state}_2(t) = \control(t), \quad \mid \control(t) \mid \le 1. \nonumber
	\label{eq:double_integ_first_ord}
\end{align}
%
The \textit{reachability problem is to address the possibility of reaching all points in the state space in a \textbf{transient} manner}. %That is, we would like to find the set of points on the state space, at a particular time step, such that we can bring the system to equilibrium, say, $\left(0, 0\right)$ -- and once we reach equilibrium,  we would like to remain on these states for all future times. 
Therefore, we set the running cost to zero, so that the Hamiltonian is $H = p_1 \dot{\state}_1 + p_2 \dot{\state}_2$. The necessary optimality condition stipulates that the minimizing control law is $\control(t) = -\text{ sign }(p_2(t))$. On a finite time interval, say, $t \in [t_0, t_f]$, the time-optimal $\control(t)$ is a constant $k$ so that for initial conditions $\state_1(t_0) = \bm{\xi}_1$ and $\state_2(t_0) = \bm{\xi}_2$, it can be  verified that the state trajectories obey the relation
%
\begin{align}
	\state_1(t) = \bm{\xi}_1 + \frac{1}{2} k  \left(\state_2^2 - \bm{\xi}_2^2\right), \,\text{where, } t = k \left(\state_2(t) - \bm{\xi}_2\right).
\end{align}
%of this the terminal state of a chosen value function on the origin of the 

The trajectories traced out over a finite time horizon $t=[-1, 1]$ on a state space and under the control laws $\control(t) = \pm 1$ is depicted in \autoref{fig:ttr_trajectories}. 
 %
\begin{figure}[tb!]
	\centering
	\includegraphics[width=\columnwidth]{figures/doub_int_trajos.jpg}
	\caption{State trajectories of the double integral plant. The solid curves are trajectories generated for $\control=+1$ while the dashed curves are trajectories for $\control = -1$.}
	\label{fig:ttr_trajectories}
\end{figure}
%
The curves with arrows that point upwards denote trajectories under the control law  $\control=+1$; call these trajectories $\switchcurve_+$; while the  trajectories with dashed curves and downward pointing arrows were executed under $\control=-1$; call these trajectories $\switchcurve_-$. The time to go from any point on any of the intersections to the origin on the state trajectories of \autoref{fig:ttr_trajectories} is our approximation problem.  This minimum time admits an analytical solution~\cite{AthansFalb} given by
%
\begin{align}
	t^\star(\state_1, \state_2) = %t^\star = 
	\begin{cases}
		\state_2 + \sqrt{4 \state_1 + 2 \state_2^2} \, &\text{if } \, \state_1 > \dfrac{1}{2} \state_2 |\state_2| \\
		%
		-\state_2 + \sqrt{-4 \state_1 + 2 \state_2^2} \, &\text{if }  \, \state_1 < -\dfrac{1}{2} \state_2 |\state_2| \\
		%
		|\state_2| &\text{if } \state_1 = \dfrac{1}{2} \state_2 |\state_2|.
	\end{cases}
\end{align}


Let us define $R_+$ as the portions of the state space above the curve $\switchcurve$ and $R_-$ as the portions of the state space below the curve $\switchcurve$.  The confluence of the locus of points on $\switchcurve_+$ and $\switchcurve_-$ is the switching curve, depicted on the left inset of \autoref{fig:attr}, and given as
%
\begin{align}
	\switchcurve \triangleq \switchcurve_+ \cup \switchcurve_- &= \left\{(\state_1, \state_2): \state_1 = \frac{1}{2}\state_2 \mid \state_2 \mid \right\}.
\end{align}
%
\begin{figure}[tb!]
	\centering
%	\begin{minipage}[b]{.24\textwidth}
%		\includegraphics[width=1.0\textwidth, height=1.0\textwidth]{figures/switching_curve.jpg}
%	\end{minipage}
	%\hfill 
	\begin{minipage}[b]{.5\textwidth}
		\includegraphics[width=1.0\textwidth]{figures/attr.jpg}
		%\subcaption{\footnotesize Fabric in uncured silicone.}
	\end{minipage}
	\caption{\footnotesize %(L) Switching curve for the double integral plant. (R) 
		Analytical time to reach the origin on the state grid, $\left(\reline \times \reline\right)$; the switching curve corresponds to the bright orange coloration for states on $(0,0)$.}
	\label{fig:attr}
\end{figure}
%
%\begin{figure}[tb!]
%\centering 
%\includegraphics[width=\columnwidth]{figures/attr.jpg}
%\caption{Analytical time to reach the origin of the state space on $\left(\reline \times \reline\right)$; the switching curve corresponds to the bright orange coloration for states on $(0,0)$.}
%\label{fig:attr}
%\end{figure}
%
%
We now state the \textbf{time-optimal control problem}: \textit{The control problem is to find the control law that forces \eqref{eq:double_integ_first_ord} to the origin $\left(0,0\right)$ in the \textbf{shortest possible time}}. The time-optimal control law, $\control^\star$, that solves this problem is unique and is
%
\begin{align}
	\control^\star &= \control^\star(\state_1, \state_2) = +1 \quad \forall \left(\state_1, \state_2\right) \in \switchcurve_+ \cup \reline_+ \nonumber \\
	%
	\control^\star &= \control^\star(\state_1, \state_2) = -1 \quad \forall \left(\state_1, \state_2\right) \in \switchcurve_- \cup \reline_-  \\
	%
	\control^\star &= -\text{sgn} \left\{\state_2\right\} \quad \forall \left(\state_1, \state_2\right) \in \switchcurve. \nonumber
\end{align}
%
The minimum cost for the problem at hand is the minimum time for states $(\state_1, \state_2)$ to reach the origin $(0, 0)$, defined as 
%
\begin{align}
	V^\star(\state, t) = t^\star(\state_1, \state_2)
\end{align}
%
with the associated terminal value
%
\begin{align}
	-\dfrac{\partial \valuefunc^\star (\state, t)}{\partial t} = \hamfunc\left(t, \state, \dfrac{\partial \valuefunc^\star (\state, t)}{\partial t}, \control \right)\bigg\rvert_{\substack{\state=\state^\star\\ \control=\control^\star}}
\end{align}
%
where 
%
\begin{align}
	\hamfunc(t; \state, \control, p_1, p_2) = \state_2(t) p_1(t) + \control(t)p_2(t)
\end{align}
%
and 
%
\begin{align}
	p_1 = \dfrac{\partial t^\star}{\partial \state_1}, \,\, 
	p_2 = \dfrac{\partial t^\star}{\partial \state_2}
\end{align}
%
so that the HJ equation is
%
\begin{align}
	\dfrac{\partial t^\star}{\partial t} &+ \state_2 \dfrac{\partial t^\star}{\partial \state_1} - \dfrac{\partial t^\star}{\partial \state_2} = 0 & \text{if }  \, \state_1 > -\dfrac{1}{2} \state_2 |\state_2|  \nonumber \\
	%
	\dfrac{\partial t^\star}{\partial t} &+ \state_2 \dfrac{\partial t^\star}{\partial \state_1} + \dfrac{\partial t^\star}{\partial \state_2} = 0  & \text{if }  \, \state_1 < -\dfrac{1}{2} \state_2 |\state_2|  \nonumber \\
	%
	\dfrac{\partial t^\star}{\partial t} &+ \state_2 \dfrac{\partial t^\star}{\partial \state_1} - \text{sgn}\{\state_2\} \dfrac{\partial t^\star}{\partial \state_2} = 0  & \text{if }  \, \state_1 = -\dfrac{1}{2} \state_2 |\state_2|.  
	\label{eq:dint_time_optimal}
\end{align}
%
%or can be verified to be 
%
%\begin{align}
%	\begin{align}
%		\dfrac{2 \state_2}{\sqrt{4\state_1 + 2 \state_2^2}}-1-\dfrac{2 \state_2}{\sqrt{4\state_1 + 2 \state_2^2}}
%	\end{align}
%\end{align}

We compare our approximated terminal value solution using our proposal against (i) the numerical solution found via level sets methods~\cite{LevelSetsBook}  and (ii) the analytical solution of the \textit{time to reach (TTR) the origin} problem.

\begin{figure}[tb!]
	\centering
	\begin{minipage}[b]{.5\textwidth}
		\includegraphics[width=.8\textwidth]{figures/dint_ttr_0.25.jpg}
	\end{minipage}
	%	
	\begin{minipage}[b]{.5\textwidth}
		\includegraphics[width=.8\textwidth]{figures/dint_ttr_1.5.jpg}
	\end{minipage}
	%
	\begin{minipage}[b]{.5\textwidth}
		\includegraphics[width=.8\textwidth]{figures/dint_ttr_2.0.jpg}
	\end{minipage}
	\caption{\footnotesize Time to reach the origin at different integration steps. Left: Analytic Time to Reach the Origin. Right: Lax-Friedrichs Approximation to Time to Reach the Origin.}
	\label{fig:ttr_ls}
\end{figure}

A point $(\state_1, \state_2)$ on the state grid belongs to the set of states $S(t^\star)$ from which it can be forced to the origin $(0, 0)$ in the same minimum time $t^\star$. We call the set $S(t^\star)$ the minimum isochrone\footnote{These are the isochrones of the system -- akin to the isochrone map used in geography, hydrology and transportation planning for depicting areas of equal travel time to a goal state.}. The level sets of \eqref{eq:dint_time_optimal} correspond to the \textit{isochrones} of the system as illustrated in \autoref{fig:ttr_ls}. From the results shown in \autoref{fig:ttr_ls}, we see that the the approximation to the isochrones by a Lax-Fridrichs scheme (right insets in the figure) are very similar. The expansion in the sets is because we overapproximate the reachable sets at each step of the integration scheme.

\subsection{Dubins Car Dynamics in Absolute coordinates}

\begin{itemize}
	\item All vehicles have identical dynamics
	%
	\item anisotropy is reinforced by having one player being a pursuer in a local group and all other agents being evaders
	%
	\item It is assumed that the roles of P and E do not change during the game, so that, when capture can occur, a necessary condition to be satisfied by the saddle-point controls of the players is the Hamiltonian (which can be derived as in Ref. 1) ~\cite{Merz1972}
	%
	\item controls are normalized turn rates of P and all other E's
	%
	\item we turn off the capture parameter by ensuring players' speeds and maximum turn radius are equal in a flock
	%
	\item  to do this, make initial velocities parallel so that the equations of relative motion mean that the Evaders can maintain the initial separation forever by simply duplicating the strategy of the P. The barrier of a local flock is thus closed, so that the game of kind is ensued with finding the determination of the closed surface.
	%
	\begin{itemize}
		\item \comment{Question is how to vary these strategies so that each we can vary anisotropy within a flock }
	\end{itemize}
	%
	\item  see solution to the homicidal chauffeur game in 9.1 of Isaacs
\end{itemize}


\section{Discussions}

\todo{Relation with Game Theory}

\todo{Relation with Reachability analysis}
