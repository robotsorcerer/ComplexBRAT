\section{Problem Statement}
\label{sec:prob_state}
%
Our goal is to find a \textit{balanced} reduced-order model that utilize a norm induced on a Hilbert space in order to produce high energy  decomposition modes  that admit a  solution to the original p.d.e problem \textit{in at least a weak sense}. We emphasize the notion of decomposing a value function so that only highly controllable and observable modes are selected. In order words, given the value function $\valuefunc(\cdot)$, a reduced value function truncated at an $r$th most-energetic mode \ie $\valuefunc_r(\cdot)$  is chosen so that
%
\begin{align}
	\lim_{t\rightarrow \infty} \valuefunc_r(\state, t) \rightarrow \valuefunc(\state, t) .
	\label{eq:value_statement}
\end{align}
%
For technological reasons, $\valuefunc_r(\state, t) $ must be derived from $\valuefunc(\state, t)$ in order to have physically realizable control laws. Thus, we seek to find the best approximation to $\valuefunc(\state, t)$ whilst preserving 
%
\begin{inparaenum}[(a)]
	\item most of the dynamic effects of the original nonlinear value function;
	%
	\item original system's attractor dynamics;
	%
	\item inherent stochasticity that may be in the original system.
\end{inparaenum}