\section{Introduction}

% motivation for the work 
% why does this work matter in the grand scheme
Recent developments in cyberphysical systems (CPS) have created complex entanglements with interactions that are difficult to analyze.
The ``physical" and ``cyber" couplings of such systems due to complex interconnection of control systems, sensors, and software make planning and executing in real-time, safety-critical scenarios like collision avoidance in uneven terrains, or sensing efficiently in the presence of multiple agents\textemdash all require deep integration and the any actions of system components must be planned meticulously. 
Therefore, the safety analysis of combined CPS systems in the presence of sensing, control, and learning becomes timely and crucial.
Differential optimal control theory and games offer a powerful paradigm for resolving the safety of multiple agents interacting over a shared space. 
Both problems rely on a resolution of the Hamilton-Jacobi-Bellman (HJB) or the Hamilton-Jacobi-Isaacs (HJI) equation in order to solve the control problem~\cite{++}.  
As HJ-type equations have no classical solution for almost all \textit{practical} problems, stable numerical and computational methods need to be brought to bear in order to produce solutions with (approximately) optimal guarantees. 

% what has been done and why does it suck
With essentially non-oscillating (ENO)~\cite{OsherShuENO} Lax-Friedrichs~\cite{CrandallLaxFriedrichs} schemes applied to numerically resolve HJ Hamiltonians~\cite{Evans1984}, we can now obtain unique (viscosity) solutions to HJ-type equations with high accuracy and precision \textit{on a mesh}. 
Employing meshes for resolving inviscid Euler equations whose solutions are the derivatives of HJ equations, these methods scale exponentially with state dimensions, making them ineffective for complex systems\textemdash a direct consequence of \textit{curse of dimensionality}~\cite{Bellman1957}. 
Truncated power series methods~\cite{Jacobson1968new, JacobsonMayne, DenhamDDP, TodorovCDC} are successive approximations of HJ value functions; however, these limit the stability region of the resulting approximate controller, and require a careful tuning of the approximate controller such that it has a direct effect on the original optimal control problem.  
In addition, stability is not easily guaranteed for series approximated HJ value functions where it is generally assumed that the highest-ordered terms in the series truncation dominate neglected higher-order terms.

% what are we going to do about it that make it better
\paragraph{Contributions} Therefore in subject matter and emphasis, this paper reflects the influences described in the foregoing. 
As a result, we focus on computational techniques because almost all \textit{practical} problems cannot be analytically resolved. 
To analyze safety, we cast our problem formulation within the framework of \textit{Cauchy-type} HJ equations~\cite{Crandall1983viscosity}, and we specifically resolve the scalable safety problem by solving the terminal value in the HJ PDE within the framework of Mitchell's \textit{robustly controlled backward reachable tubes}~\cite{Mitchell2020}. %Extensions to \textit{Dirichlet-type} HJ equations are straightforward.  
In this sentiment, new computational techniques are introduced including
%
\begin{inparaenum}[(i)]
	\item iterative Galerkin approximation of large value functions;
	%
	\item finite difference approximation schemes with error estimates (essentially, an extension of ~\cite{Crandall1984} on reduced Hilbertian spaces); and
	%
	\item  analytic saddle solutions to approximated HJ value functions.
\end{inparaenum} 
%
to synthesize approximately optimal control laws (essentially, saddle-point solutions) 
%\todo{with stability guarantees} for resolving the terminal value in the viscosity solutions to HJI value functions.

% structure of paper
% TODO: update this when paper is finalized. 
The rest of this paper is organized as follows: \autoref{sec:related} describes the concepts and \autoref{sec:prelim} topics that we will build upon in describing our proposal in \autoref{sec:methods}; we present results and insights from experiments in \autoref{sec:results}. 
We conclude the paper in \autoref{sec:conclude}. This work is the first to systematically provide a rational incremental decomposition scheme that provides approximation guarantees on regions of the state space where approximate HJ control laws are valid as well as provide a rational analysis for high-dimensional verification of nonlinear systems with guarantees.  %\todo{Add more stuff that encompasses the problems we address in this paper here.}     