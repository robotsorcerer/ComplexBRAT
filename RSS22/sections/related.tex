\section{Related Work}

\paragraph{Dynamic Programming and Two-Person Games.}
The formal relationships between the dynamic programming (DP) optimality condition for the \textit{value} in differential two-person zero-sum games, and the solutions to PDEs that solve ``min-max" or ``max-min" type nonlinearity (the Isaacs' equation) was presented in~\citet{Isaacs1965}. 
Essentially, Isaacs' claim was that if the \textit{value} functions are smooth enough, then they solve certain first-order partial differential equations (PDE) problems with  ``max-min" or ``min-max"-type nonlinearity.  
However, the DP value functions are seldom regular enough to admit a solution in the classical sense.  
``Weaker" solutions on the other hand~\citet{Lions1982, Evans1984, Crandall1984, CrandallLaxFriedrichs, Souganidis} provide generalized ``viscosity" solutions to HJ PDEs under relaxed regularity conditions; these viscosity solutions are not necessarily differentiable anywhere in the state space, and the only regularity prerequisite in the definition is continuity~\citet{Crandall1983viscosity}. 
However, wherever they are differentiable, they satisfy the  upper and lower values of HJ PDEs in a classical sense. 
Thus, they lend themselves well to many real-world problems existing at the interface of discrete, continuous, and hybrid systems~\citet{LygerosReachability, OsherFronts, Mitchell2020, Souganidis, Mitchell2005}. 
Viscosity Solutions to \textit{Cauchy-type} HJ Equations admit usefulness in backward reachability analysis~\citet{Mitchell2005}. 
In scope and focus, this is the bulwark upon which we build our formulation in this paper.


\paragraph{Reachability for Systems Verification.}
Reachability analysis is one of many verification methods that allows us to reason about (cpntrol-affine) dynamical systems. 
%For CPS systems, the scalability of existing verification methods is a requirement for the proper verification of complex systems.  
The verification problem may consist in finding a \textit{set of reachable states} that lie along the trajectory of the solution to a first order nonlinear partial differential equation that originates from some initial state $\state_0 = \state(0)$ up to a specified time bound, $t=t_f$. 
\textit{From a set of initial and unsafe state sets, and a time bound, the time-bounded safety verification problem is to determine if there is an initial state and a time within the bound that the solution to the PDE enters the unsafe set}.
Reachability could be analyzed in a 
%
\begin{inparaenum}[(i)]
	\item \textit{forward} sense, whereupon system trajectories are examined to determine if they enter certain states from an \textit{initial set};
	%
	\item \textit{backward} sense, whereupon system trajectories are examined to determine if they enter certain \textit{target sets};
	%
	\item \textit{reach set} sense, in which they are examined to see if states reach a set at a \textit{particular time}; or
	%
	\item \textit{reach tube} sense, in which they are evaluated that they reach a set at a point \textit{during a time interval}.	
\end{inparaenum} 

Backward reachable sets (BRS) or tubes (BRTs) are popularly analyzed as a game of two vehicles with non-stochastic dynamics~\citet{Merz1972}. 
Such BRTs possess discontinuity at cross-over points (which exist at edges) on the surface of the  tube, and may be non-convex. 
Therefore, treating the end-point constraints under these discontinuity characterizations need careful consideration and analysis when switching control laws if the underlying PDE does not have continuous partial  derivatives \todo{(we discuss this further in \autoref{sec:methods})}. 

\paragraph{Global Mesh-based Methods and Up-Scaling Reachability Analysis}
Consider a reachability problem defined in a space of dimension $D=12$ based on the non-incremental time-space discretization of each space coordinate. For $N=100$ nodes, the total nodes required is $10^{120}$ on the volumetric grid\footnote{Whereas, there are only $10^{97}$ baryons in the observable universe (excluding dark matter)!}. The curse of dimensionality~\citet{Bellman1957} greatly incapacitates current uniform grid discretization methods for guaranteeing the robustness of backward reachable sets (BRS) and tubes (BRTs)~\citet{Mitchell2005} of complex systems. 
Recent works have started exploring scaling up the Cauchy-type HJ problem for guaranteeing safety of higher-dimensional physical systems: the authors of~\citet{Bajcsy} provide local updates to BRS in  unknown static environments with obstacles that may be unknown \textit{a priori} to the agent; using standard meshing techniques for time-space uniform discretization over the entire physical space, and only updating points traversed locally, a safe navigation problem was solved in an environment assumed to be static. 
This makes it non-amenable to \textit{a priori} unknown \textit{dynamic} environments where the optimal value to the min-max HJ problem may need to be adaptively updated based on changing dynamics. 

In ~\citet{SylviaScalability}, the grid was naively refined along the temporal dimension, leveraging local decomposition schemes together with warm-starting optimization of the value function from previous solutions in order to accelerate learning for safety under the assumption that the system is either completely decoupled, or coupled over so-called ``self-contained subsystems".  While the empirical results of~\citet{Bansal} demonstrate the feasibility of optimizing for the optimal value function in backward reachability analysis for up to ten dimensions for a system of Dubins vehicles, there are no guarantees that are provided. An analysis exists for a 12 dimensional systems~\citet{KaynamaScalable} with up to a billion data points in the state space, that generates robustly optimal trajectories. However, this is restricted to {linear systems}. Other associated techniques scale reachability with function approximators~\citet{FisacTAC, FisacICRA} in a reinforcement learning framework; again these methods lose the hard safety guarantees owing to the approximation in value function space.  

